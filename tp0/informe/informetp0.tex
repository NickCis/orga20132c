%%%%%%%%%%%%%%%%%%%%%%%%%%%%%%%%%%%%%%%%%%%%%%%%%%%%%%%%%%%%%%%%%%%%%%%%%%%%%%%
% Definici�n del tipo de documento.                                           %
% Posibles tipos de papel: a4paper, letterpaper, legalpapper                  %
% Posibles tama�os de letra: 10pt, 11pt, 12pt                                 %
% Posibles clases de documentos: article, report, book, slides                %
%%%%%%%%%%%%%%%%%%%%%%%%%%%%%%%%%%%%%%%%%%%%%%%%%%%%%%%%%%%%%%%%%%%%%%%%%%%%%%%
\documentclass[a4paper,10pt]{article}


%%%%%%%%%%%%%%%%%%%%%%%%%%%%%%%%%%%%%%%%%%%%%%%%%%%%%%%%%%%%%%%%%%%%%%%%%%%%%%%
% Los paquetes permiten ampliar las capacidades de LaTeX.                     %
%%%%%%%%%%%%%%%%%%%%%%%%%%%%%%%%%%%%%%%%%%%%%%%%%%%%%%%%%%%%%%%%%%%%%%%%%%%%%%%

% Paquete para inclusi�n de gr�ficos.
\usepackage{graphicx}
% \end{lstlisting}

% Paquete para control de la flotabilidad de las imagenes
\usepackage{float}
% Paquete para definir la codificaci�n del conjunto de caracteres usado
% (latin1 es ISO 8859-1).
\usepackage[latin1]{inputenc}

% Paquete para definir el idioma usado.
\usepackage[spanish,activeacute]{babel}

% Paquetes para manejar lenguajes
\usepackage{listings,multicol}

%%%%%%%%%%%%%%%%%%%%%%%%%%%%%%%%%%%%%%%%%%%%%%%%%%%%%%%%%%%%%%%%%%%%%%%%%%%%%%%%%%%%
%%%%%%%%%%%%%%%%%%%%%%%%%%%%%%%%DECLARACIONES%%%%%%%%%%%%%%%%%%%%%%%%%%%%%%%%%%%%%%%
%%%%%%%%%%%%%%%%%%%%%%%%%%%%%%%%%%%%%%%%%%%%%%%%%%%%%%%%%%%%%%%%%%%%%%%%%%%%%%%%%%%%
\lstdefinestyle{BashInputStyle}{
  language=bash,
  basicstyle=\small\sffamily,
  numbers=left,
  numberstyle=\tiny,
  numbersep=3pt,
%    frame=tb, %DESCOMENTAR PARA PONERLE LINEAS ARRIBA Y ABAJO
  columns=fullflexible,
  linewidth=0.9\linewidth,
  xleftmargin=0.1\linewidth
}% \end{lstlisting}

\newcommand*{\Package}[1]{\texttt{#1}}%

\lstdefinestyle{customasm}{
    belowcaptionskip=1\baselineskip,
  frame=L,
  xleftmargin=\parindent,
  language=[x86masm]Assembler,
  basicstyle=\scriptsize\ttfamily,
  commentstyle=\itshape\color{purple!40!black},
}

%%%%%%%%%%%%%%%%%%%%%%%%%%%%%%%%%%%%%%%%%%%%%%%%%%%%%%%%%%%%%%%%%%%%%%%%%%%%%%%%%%%%
%%%%%%%%%%%%%%%%%%%%%%%%%%%%%%%%FIN DECLARACIONES %%%%%%%%%%%%%%%%%%%%%%%%%%%%%%%%%%
%%%%%%%%%%%%%%%%%%%%%%%%%%%%%%%%%%%%%%%%%%%%%%%%%%%%%%%%%%%%%%%%%%%%%%%%%%%%%%%%%%%%

% T�tulo principal del documento.
\title{		\textbf{Trabajo pr'actico 0}}

% Informaci�n sobre los autores.
\author{
	Nicolas Cisco, \textit{Padr�n Nro. 94.173}                     \\
		\texttt{ ncis20@gmail.com }                                              \\
	Rodrigo Burdet, \textit{Padr�n Nro. 93.440}                     \\
		\texttt{ rodrigoburdet@gmail.com }                                              \\
	Nicolas Calvo, \textit{Padr�n Nro. 00.000}                     \\
		\texttt{ nicolas.g.calvo@gmail.com }                                              \\[2.5ex]
	\normalsize{Grupo Nro.  - 2do. Cuatrimestre de 2013}                       \\
	\normalsize{66.20 Organizaci�n de Computadoras}                             \\
	\normalsize{Facultad de Ingenier�a, Universidad de Buenos Aires}            \\
}
\date{}



\begin{document}

% Inserta el t�tulo.
\maketitle

% Quita el n�mero en la primer p�gina.
\thispagestyle{empty}

% Resumen
\begin{abstract}
Este trabajo pr'actico sirve como introducci'on a las herramientas b'asicas que se usar'an m'as adelante. El programa escrito en C genera una imagen de un tablero de ajedrez seg'un los argumentos que sean pasados.
El formato gr'afico a utilizar fue PGM o \textit{portable grey map}.
\end{abstract}


\section{Introducci�n}

El formato gr'afico PGM sirve para describir im'agenes digitales monocrom'aticas. Siendo un tablero de ajedrez un claro ejemplo de una imagen monocrom'atica se busca dibujar uno de dimensiones especificadas a trav'es de argumentos ingresador por consola. La salida del programa contiene el tablero en formato PGM.
El programa (que fue hecho en {\bf ANSI C}) fue emulado en una m'aquina MIPS32 y habiendo pasado los casos de prueba se obtuvo el c'odigo assembly generado por el compilador {\bf GCC}

\section{Desarrollo}

El trabajo pr'actico fue dividido en dos secciones: 
\begin{itemize}
 \item Parser de argumentos (TDA)
 \item Main
\end{itemize}

\subsection{Parser}
Para crear el TDA parseador de argumentos se le pasa la cantidad m'axima de argumentos que el programa puede llegar a recibir, internamente se guarda un vector de estructuras \emph{TArg}. Cada uno de estos \emph{TArg} es un posible argumento, esta estructura guarda una funci'on parseadora de char* provenientes de argv a un valor, un char representando el nombre corto (e.g: o de -o), un char* representando el nombre largo (e.g: output de --output), un void* representando un puntero a un valor por defecto, y dos valores que se asignan despu'es de parsear argv (si es que se realiza alg'un match), un entero para saber si se encontr'o el valor y otro char* correspondiente al string del argv que se us'o como argumento (e.g: -o output.pgm, se guardar'a output.pgm).
En el .h se describe que hacen las primitivas, a modo de resumen, se carga el array de argumentos usando \textit{ParseArg\_addArg}. Despues, se llama a \textit{ParseArg\_parse(argc, argv)} que recorre el vector argv buscando los par'ametros. Para obtener el valor de un argumento, se llama a \textit{ParseArg\_getArg(nombre\_corto)} siendo nombre corto  un char, esto devuelve un puntero al dato o null, internamente, se busc'o en el vector de \textit{TArg} aquel con nombre corto igual al pasado por argumento .Si \textit{arg $\rightarrow$ encontre == 0}, se devuelve el valor por defecto, sino, si \textit{arg $\rightarrow$ func == NULL} se devuelve un 1 casteado a void*, si \textit{arg $\rightarrow$buf == NULL} se devuelve NULL, sino, se devuelve \textit{arg $\rightarrow$func(arg$\rightarrow$buf)} (buf siempre es una cadena de caracteres finalizada en 0).

\subsection{Main}
Para imprimir la matriz, primero se imprime la cabecera, y dsp se usan 4 ciclos while.
El primero es para la cantidad de filas (siempre van a ser 8), el segundo es para la repetici'on de filas (\textit{c\_h} siempre es un entero, por ser divisi'on de enteros, como \textit{c\_h = pixeles\_altura / 8}, si \textit{pixeles\_altura} = 8, solo se repetir'a una vez cada linea, si \textit{pixeles\_altura} =16, 2, y asi sucecivamente), el siguiente while es para la cantidad de columnas, \textit{c\_w} es lo mismo que \textit{c\_h}, solo qe para el ancho. Se usa la variable uno para ir alternando fila a fila si se empieza con 1 o 0, se usa \emph{j mod 2} para cambiar el n'umero qe se imprime columna a columna.


\section{Pruebas}
Prueba de impres'ion de mensaje de ayuda:
\begin{lstlisting}[style=BashInputStyle]
$ ./tp0 -h
Usage:
        ./tp0 -h
        ./tp0 -V
        ./tp0 [options]
Options:
        -V, --version           Print version and quit.
        -h, --help              Print this information.
        -r, --resolution        Set bitmap resolution to WxH pixels.
        -o, --output            Path to output file.
Examples:
        ./tp0 -o output.pgm
        ./tp0 -r 800x600 -o file.pgm
\end{lstlisting}

Primer ejemplo con salida por por stdout:
\begin{lstlisting}[style=BashInputStyle]
$ ./tp0 -r 8x8 -o -
P2
8
8
1
1 0 1 0 1 0 1 0
0 1 0 1 0 1 0 1
1 0 1 0 1 0 1 0
0 1 0 1 0 1 0 1
1 0 1 0 1 0 1 0
0 1 0 1 0 1 0 1
1 0 1 0 1 0 1 0
0 1 0 1 0 1 0 1
\end{lstlisting}

Segundo ejemplo con salida por por stdout:
\begin{lstlisting}[style=BashInputStyle]
$ ./tp0 -r 16x8 -o -
P2
16
8
1
1 1 0 0 1 1 0 0 1 1 0 0 1 1 0 0
0 0 1 1 0 0 1 1 0 0 1 1 0 0 1 1
1 1 0 0 1 1 0 0 1 1 0 0 1 1 0 0
0 0 1 1 0 0 1 1 0 0 1 1 0 0 1 1
1 1 0 0 1 1 0 0 1 1 0 0 1 1 0 0
0 0 1 1 0 0 1 1 0 0 1 1 0 0 1 1
1 1 0 0 1 1 0 0 1 1 0 0 1 1 0 0
0 0 1 1 0 0 1 1 0 0 1 1 0 0 1 1
\end{lstlisting}

Ejemplo guardando resultado en archivo:
\begin{lstlisting}[style=BashInputStyle]
$ ./tp0 -r 16x16 -o /tmp/test
$ cat /tmp/test
P2
16
16
1
1 1 0 0 1 1 0 0 1 1 0 0 1 1 0 0
1 1 0 0 1 1 0 0 1 1 0 0 1 1 0 0
0 0 1 1 0 0 1 1 0 0 1 1 0 0 1 1
0 0 1 1 0 0 1 1 0 0 1 1 0 0 1 1
1 1 0 0 1 1 0 0 1 1 0 0 1 1 0 0
1 1 0 0 1 1 0 0 1 1 0 0 1 1 0 0
0 0 1 1 0 0 1 1 0 0 1 1 0 0 1 1
0 0 1 1 0 0 1 1 0 0 1 1 0 0 1 1
1 1 0 0 1 1 0 0 1 1 0 0 1 1 0 0
1 1 0 0 1 1 0 0 1 1 0 0 1 1 0 0
0 0 1 1 0 0 1 1 0 0 1 1 0 0 1 1
0 0 1 1 0 0 1 1 0 0 1 1 0 0 1 1
1 1 0 0 1 1 0 0 1 1 0 0 1 1 0 0
1 1 0 0 1 1 0 0 1 1 0 0 1 1 0 0
0 0 1 1 0 0 1 1 0 0 1 1 0 0 1 1
0 0 1 1 0 0 1 1 0 0 1 1 0 0 1 1
\end{lstlisting}

En la figura~\ref{fig001} se muestra el archivo {\bf file.pmg} cuando se llama al programa :
\begin{lstlisting}[style=BashInputStyle]
    $./tp0 -r 800x600 -o file.pgm
\end{lstlisting}

% Inclusi�n de una imagen en formato EPS (Encapsulated Postscript).

\begin{figure}[H]
\begin{center}
\includegraphics[width=0.5\textwidth]{file}
\end{center}
\caption{file.pgm (800x600)} \label{fig001}
\end{figure}

%\section{C'odigo}
%\twocolumn
%% \begin{lstlisting}[numbers=left,xleftmargin=3em, multicols=2]
%\lstinputlisting[style=customasm]{main.s}



\onecolumn

\section{Conclusiones}
Se present� un modelo para que los alumnos puedan tomar como referencia en la redacci�n de sus informes de trabajos pr�cticos.


% Citas bibliogr�ficas.
\begin{thebibliography}{99}

\bibitem{INT06} Intel Technology \& Research, ``Hyper-Threading Technology,'' 2006, http://www.intel.com/technology/hyperthread/.

\bibitem{HEN00} J. L. Hennessy and D. A. Patterson, ``Computer Architecture. A Quantitative
Approach,'' 3ra Edici�n, Morgan Kaufmann Publishers, 2000.

\bibitem{LAR92} J. Larus and T. Ball, ``Rewriting Executable Files to Mesure Program Behavior,'' Tech. Report 1083, Univ. of Wisconsin, 1992.

\end{thebibliography}

\end{document}
