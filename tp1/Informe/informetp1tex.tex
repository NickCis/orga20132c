%%%%%%%%%%%%%%%%%%%%%%%%%%%%%%%%%%%%%%%%%%%%%%%%%%%%%%%%%%%%%%%%%%%%%%%%%%%%%%%
% Definici�n del tipo de documento.                                           %
% Posibles tipos de papel: a4paper, letterpaper, legalpapper                  %
% Posibles tama�os de letra: 10pt, 11pt, 12pt                                 %
% Posibles clases de documentos: article, report, book, slides                %
%%%%%%%%%%%%%%%%%%%%%%%%%%%%%%%%%%%%%%%%%%%%%%%%%%%%%%%%%%%%%%%%%%%%%%%%%%%%%%%
\documentclass[a4paper,10pt]{article}


%%%%%%%%%%%%%%%%%%%%%%%%%%%%%%%%%%%%%%%%%%%%%%%%%%%%%%%%%%%%%%%%%%%%%%%%%%%%%%%
% Los paquetes permiten ampliar las capacidades de LaTeX.                     %
%%%%%%%%%%%%%%%%%%%%%%%%%%%%%%%%%%%%%%%%%%%%%%%%%%%%%%%%%%%%%%%%%%%%%%%%%%%%%%%

% Paquete para inclusi�n de gr�ficos.
\usepackage{graphicx}
% \end{lstlisting}

% Paquete para control de la flotabilidad de las imagenes
\usepackage{float}
% Paquete para definir\lstdefinestyle{BashInputStyle}{
% \end{lstlisting} la codificaci�n del conjunto de caracteres usado
% (latin1 es ISO 8859-1).
\usepackage[latin1]{inputenc}

% Paquete para definir el idioma usado.
\usepackage[spanish,activeacute]{babel}

% Paquetes para manejar lenguajes
\usepackage{listings,multicol}

%%%%%%%%%%%%%%%%%%%%%%%%%%%%%%%%%%%%%%%%%%%%%%%%%%%%%%%%%%%%%%%%%%%%%%%%%%%%%%%%%%%%
%%%%%%%%%%%%%%%%%%%%%%%%%%%%%%%%DECLARACIONES%%%%%%%%%%%%%%%%%%%%%%%%%%%%%%%%%%%%%%%
%%%%%%%%%%%%%%%%%%%%%%%%%%%%%%%%%%%%%%%%%%%%%%%%%%%%%%%%%%%%%%%%%%%%%%%%%%%%%%%%%%%%
\lstdefinestyle{BashInputStyle}{
  language=bash,
  basicstyle=\small\sffamily,
  numbers=left,
  numberstyle=\tiny,
  numbersep=3pt,
%    frame=tb, %DESCOMENTAR PARA PONERLE LINEAS ARRIBA Y ABAJO
  columns=fullflexible,
  linewidth=0.9\linewidth,
  xleftmargin=0.1\linewidth
}% \end{lstlisting}

\newcommand*{\Package}[1]{\texttt{#1}}%

\lstdefinestyle{customasm}{
    belowcaptionskip=1\baselineskip,
  frame=L,
  xleftmargin=\parindent,
  language=[x86masm]Assembler,
  basicstyle=\scriptsize\ttfamily,
  commentstyle=\itshape\color{purple!40!black},
}

%%%%%%%%%%%%%%%%%%%%%%%%%%%%%%%%%%%%%%%%%%%%%%%%%%%%%%%%%%%%%%%%%%%%%%%%%%%%%%%%%%%%
%%%%%%%%%%%%%%%%%%%%%%%%%%%%%%%%FIN DECLARACIONES %%%%%%%%%%%%%%%%%%%%%%%%%%%%%%%%%%
%%%%%%%%%%%%%%%%%%%%%%%%%%%%%%%%%%%%%%%%%%%%%%%%%%%%%%%%%%%%%%%%%%%%%%%%%%%%%%%%%%%%

% T�tulo principal del documento.
\title{		\textbf{Trabajo pr'actico 0}}

% Informaci�n sobre los autores.
\author{
	Nicolas Cisco, \textit{Padr�n Nro. 94.173}                     \\
		\texttt{ ncis20@gmail.com }                                              \\
	Rodrigo Burdet, \textit{Padr�n Nro. 93.440}                     \\
		\texttt{ rodrigoburdet@gmail.com }                                              \\
	Nicolas Calvo, \textit{Padr�n Nro. 78.914}                     \\
		\texttt{ nicolas.g.calvo@gmail.com }                                              \\[2.5ex]
	\normalsize{Grupo Nro.  - 2do. Cuatrimestre de 2013}                       \\
	\normalsize{66.20 Organizaci�n de Computadoras}                             \\
	\normalsize{Facultad de Ingenier�a, Universidad de Buenos Aires}            \\
}
\date{}




\begin{document}

% Inserta el t�tulo.
\maketitle

% Quita el n�mero en la primer p�gina.
\thispagestyle{empty}

% Resumen
\begin{abstract}
En el presente se comparan dos algoritmos de ordenamiento: \textit{Bubblesort} vs \textit{Heapsort}. Bubblesort fue solo implementado en C mientras que Heapsort fue hecho en MIPS32 y en C. El par'ametro de comparaci'on es el tiempo de ordenamiento de archivos de texto. 
Todos los archivos y c�digos fuente aqu� mencionados, as� como tambi�n el presente informe, pueden ser descargados del repositorio de trabajo \footnote{URL del Repositorio: \url{https://github.com/NickCis/orga20132c/tree/master/tp1}}.
\bigskip
\end{abstract}


\section{Introducci�n}
El ordenamiento por mont�culos (heapsort en ingl�s) es un algoritmo de ordenamiento no recursivo, no estable, con complejidad computacional \Theta(n\log n). Este algoritmo consiste en almacenar todos los elementos del vector a ordenar en un mont�culo (heap), y luego extraer el nodo que queda como nodo ra�z del mont�culo (cima) en sucesivas iteraciones obteniendo el conjunto ordenado. Basa su funcionamiento en una propiedad de los mont�culos, por la cual, la cima contiene siempre el menor elemento (o el mayor, seg�n se haya definido el mont�culo) de todos los almacenados en �l.
La Ordenaci�n de burbuja (Bubble Sort en ingl�s) es un sencillo algoritmo de ordenamiento. Funciona revisando cada elemento de la lista que va a ser ordenada con el siguiente, intercambi�ndolos de posici�n si est�n en el orden equivocado. Es necesario revisar varias veces toda la lista hasta que no se necesiten m�s intercambios, lo cual significa que la lista est� ordenada. Este algoritmo obtiene su nombre de la forma con la que suben por la lista los elementos durante los intercambios, como si fueran peque�as "burbujas". Tambi�n es conocido como el m�todo del intercambio directo. Dado que solo usa comparaciones para operar elementos, se lo considera un algoritmo de comparaci�n, siendo el m�s sencillo de implementar.
�ste algoritmo es esencialmente un algoritmo de fuerza bruta l�gica siendo su complejidad algor'imica de \Theta(n^2).

\section{Compilaci'on}
Para compilar el programa se uso un \textit{make} que ser'a presentado en el ap'endice para automatizar el proceso. Se opt'o por usar el standard 99 \textit{std99}\cite{std99} trantando a los warnings como errores y sin usar optimizaciones {\bf -O0}.
A saber: 
\begin{itemize}
 \item Antes de cada ejecuci'on hacer : {\bf make clean}
 \item Para compilar el c'odigo assembly se puede hacer: {\bf make assembly}
 \item Para compilar el codigo en C se puede hacer: {\bf make}
\end{itemize}


\section{Programa}
El programa consta de 3 secciones:
\begin{itemize}
 \item Parser de argumentos (TDA)
 \item M'etodos de ordenamiento
 \begin{itemize}
  \item Bubblesort
  \item Heapsort
 \end{itemize}
 \item Main
\end{itemize}

Cada una de ellas sera explicada a continuaci'on:

\section{Desarrollo}

El trabajo pr'actico fue dividido en dos secciones: 

\subsection{Parser}
Para crear el TDA parseador de argumentos se le pasa la cantidad m'axima de argumentos que el programa puede llegar a recibir, internamente se guarda un vector de estructuras \emph{TArg}. Cada uno de estos \emph{TArg} es un posible argumento, esta estructura guarda una funci'on parseadora de char* provenientes de argv a un valor, un char representando el nombre corto (e.g: o de -o), un char* representando el nombre largo (e.g: output de --output), un void* representando un puntero a un valor por defecto, y dos valores que se asignan despu'es de parsear argv (si es que se realiza alg'un match), un entero para saber si se encontr'o el valor y otro char* correspondiente al string del argv que se us'o como argumento (e.g: -o output.pgm, se guardar'a output.pgm).
En el .h se describe que hacen las primitivas, a modo de resumen, se carga el array de argumentos usando \textit{ParseArg\_addArg}. Despues, se llama a \textit{ParseArg\_parse(argc, argv)} que recorre el vector argv buscando los par'ametros. Para obtener el valor de un argumento, se llama a \textit{ParseArg\_getArg(nombre\_corto)} siendo nombre corto  un char, esto devuelve un puntero al dato o null, internamente, se busc'o en el vector de \textit{TArg} aquel con nombre corto igual al pasado por argumento .Si \textit{arg $\rightarrow$ encontre == 0}, se devuelve el valor por defecto, sino, si \textit{arg $\rightarrow$ func == NULL} se devuelve un 1 casteado a void*, si \textit{arg $\rightarrow$buf == NULL} se devuelve NULL, sino, se devuelve \textit{arg $\rightarrow$func(arg$\rightarrow$buf)} (buf siempre es una cadena de caracteres finalizada en 0).

\subsection{Main}
El main consta de todas las funciones auxiliares que son necesarias para los ordenadores. Estos necesitan una lista de palabras que ser'an provistas por \textit{fillWords}. A su vez, esta, necesita todas las palabras del archivo en un solo buffer, tarea de \textit{getData}.

\subsection{M'etodos de ordenamiento}
%%%%%%%%%%%
%%%%%%%%%%%%%
%%%%%%%%%%%%%
%%%%%%%%%%%%%////////////////??HACERRRRRRRRRRRRRRRRRRRRRRRRRRR


\section{Pruebas}

\section{Conclusiones}

\section{thebibliography}{99}
  \bibitem{STD99} C99 Standard, \url{http://en.wikipedia.org/wiki/C99}


\end{document}
